\chapter{Функции обработчика прерываний от системного таймера в системах разделения времени}
\section{ОС семейства Windows}
Обработчик прерывания от системного таймера \textbf{по тику} выполняет следующие задачи:
\begin{itemize}
	\item инкремент счетчика системного времени;
	\item декремент остатка кванта;
	\item декремент счетчиков времени отложенных задач;
	\item если активен механизм профилирования ядра, то инициализирует отложенный вызов обработчика ловушки профилирования ядра путем постановки в очередь DPC: обработчик ловушки профилирования регистрирует адрес команды, выполнявшейся на момент прерывания.
\end{itemize}

Обработчик прерывания от системного таймера \textbf{по главному тику} выполняет следующие задачи:
\begin{itemize}
	\item устанавливает в системе объект \grqq;событие\grqq, который ожидает диспетчер настройки баланса;
\end{itemize}

Обработчик прерывания от системного таймера \textbf{по кванту} выполняет следующие задачи:
\begin{itemize}
	\item инициализирует диспетчеризацию потоков путем постановки соответствующего объекта в очередь DPC;
\end{itemize}

\section{ОС семейства Unix/Linux}
Обработчик прерывания от системного таймера \textbf{по тику} выполняет следующие задачи:
\begin{itemize}
	\item инкремент таймеров системы;
	\item декремент счетчика времени до отправления на выполнение отложенного вызова;
	\item инкремент счетчика использования процессора;
	\item декремент кванта текущего потока.
\end{itemize}

Обработчик прерывания от системного таймера \textbf{по главному тику} выполняет следующие задачи:
\begin{itemize}
	\item выставляет флаг, указывающий на необходимость запуска обработчика отложенного вызова;
	\item пробуждает из состояния прерываемого сна системных вызовов swapper и pagedaemon. Пробуждение - регистрация отложенного вызова процедуры wakeup, которая перемещает дескрипторы процессов из списка "спящие" в очередь готовых процессов;
	\item декремент счетчика времени, оставшегося до посылки одного из следующих сигналов:
	\begin{itemize}
		\item SIGALRM - сигнал, посылаемый процессу по истечении времени, которое предварительно задано функцией alarm();
		\item SIGPROF - сигнал, посылаемый процессу по истечении времени, заданного в таймере профилирования;
		\item SIGVTALARM - сигнал, посылаемый процессу по истечении времени, заданного в "виртуальном" таймере.
	\end{itemize}
\end{itemize}

Обработчик прерывания от системного таймера \textbf{по кванту} выполняет следующие задачи:
\begin{itemize}
	\item посылает текущему процессу сигнал SIGXCPU, если он израсходовал выделенный ему квант процессорного времени.
\end{itemize}